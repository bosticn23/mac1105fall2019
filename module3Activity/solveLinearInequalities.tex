\documentclass{ximera}
\usepackage{sagetex}
%% handout
%% space
%% newpage
%% numbers
%% nooutcomes

%% You can put user macros here
%% However, you cannot make new environments

\graphicspath{{./}{module1Activity/}{module2Activity/}{module3Activity/}}

\usepackage{sagetex}
\usepackage{tikz}
\usepackage{hyperref}
\usepackage{tkz-euclide}
\usetkzobj{all}
\pgfplotsset{compat=1.7} % prevents compile error.

\tikzstyle geometryDiagrams=[ultra thick,color=blue!50!black]
 %% we can turn off input when making a master document

\outcome{Understand and solve linear inequalities.}
\author{Darryl Chamberlain Jr.}
 
\title{Objective 2 - Solve Linear Inequalities}

\begin{document}
\begin{abstract}
Solve linear inequalities. 
\end{abstract}
\maketitle

\href{https://cnx.org/contents/mwjClAV_@8.1:uIjtHMfW@9/Linear-Inequalities-and-Absolute-Value-Inequalities}{Link to section in online textbook.}

%%%%%%%%%%%%%%%%%%%%%
%%%  Objective 2  %%%
%%%%%%%%%%%%%%%%%%%%%

Now, watch \underline{\href{https://mediasite.video.ufl.edu/Mediasite/Play/ab04629334ac4cadb7a69beb924910651d}{this video}} to learn how to use the properties of inequalities to solve linear inequalities. Inequalities will come up multiple times throughout the semester in different contexts, so be sure to write out notes to yourself about their differences!

Now try to solve the different linear inequalities below. 

\begin{sagesilent}
x = var('x')
##################
def maybeMakeNegative(rational):
    maybeNegative = (-1)**ZZ.random_element(2)
    rational = maybeNegative * rational
    return rational

def createCoefficientsEasy():
    coefficients = [0, 0, 0, 0]
    while (coefficients[1] == coefficients[2]):
        coefficients[0] = maybeMakeNegative(ZZ.random_element(3, 11))
        coefficients[1] = maybeMakeNegative(ZZ.random_element(3, 11))
        coefficients[2] = maybeMakeNegative(ZZ.random_element(3, 11))
        coefficients[3] = maybeMakeNegative(ZZ.random_element(3, 11))
    return coefficients

def createIntervalSolutionEasy(coefficients, direction):
    a, b, c, d = coefficients
    equation = (a+b*x) - (c*x+d)
    endpoint = round(float(solve(equation, x)[0].rhs()), 3)
    nearEndpoint = endpoint - 1
    checkNearby = float( float(a)+float(b)*nearEndpoint - (float(c)*nearEndpoint+float(d)) )
    # Checks direction of inequality
    if direction == "less" or direction == "leq":
        if (checkNearby < 0):
            interval = [-oo, endpoint]
            whichSideFloat = "Right"
        else:
            interval = [endpoint, oo]   
            whichSideFloat = "Left"
    elif direction == "greater" or direction == "geq": 
        if (checkNearby > 0):
            interval = [-oo, endpoint]
            whichSideFloat = "Right"
        else:
            interval = [endpoint, oo]
            whichSideFloat = "Left"
    else: 
        print "You input an invalid inequality."
        interval = [0, 0]
        whichSideFloat = "Neither"
    return [interval, whichSideFloat]
######### QUESTION 3 - FORCED $[, \infty)$ Solution #########
coefficients3 = createCoefficientsEasy()
solution3 = createIntervalSolutionEasy(coefficients3, "leq")
while solution3[1] == "Right":
    coefficients3 = createCoefficientsEasy()
    solution3 = createIntervalSolutionEasy(coefficients3, "leq")
displayLeftFactor3 = coefficients3[0] + coefficients3[1] * x
displayRightFactor3 = coefficients3[2] * x + coefficients3[3]

######### QUESTION 4 - FORCED $(-\infty, ]$ Solution #########
coefficients4 = createCoefficientsEasy()
solution4 = createIntervalSolutionEasy(coefficients4, "geq")
while solution4[1] == "Left":
    coefficients4 = createCoefficientsEasy()
    solution4 = createIntervalSolutionEasy(coefficients4, "geq")
    print coefficients4, solution4
displayLeftFactor4 = coefficients4[0] + coefficients4[1] * x
displayRightFactor4 = coefficients4[2] * x + coefficients4[3]

######### QUESTION 5 - FORCED $(-\infty, )$ Solution #########
coefficients5 = createCoefficientsEasy()
solution5 = createIntervalSolutionEasy(coefficients5, "less")
while solution5[1] == "Left":
    coefficients5 = createCoefficientsEasy()
    solution5 = createIntervalSolutionEasy(coefficients5, "less")
displayLeftFactor5 = coefficients5[0] + coefficients5[1] * x
displayRightFactor5 = coefficients5[2] * x + coefficients5[3]

######### QUESTION 6 - FORCED $(, \infty)$ Solution #########
direction6 = "greater"
coefficients6 = createCoefficientsEasy()
solution6 = createIntervalSolutionEasy(coefficients6, "greater")
while solution6[1] == "Right":
    coefficients6 = createCoefficientsEasy()
    solution6 = createIntervalSolutionEasy(coefficients6, "greater")
displayLeftFactor6 = coefficients6[0] + coefficients6[1] * x
displayRightFactor6 = coefficients6[2] * x + coefficients6[3]
\end{sagesilent}

\begin{question}
$\sage{displayLeftFactor3} \leq \sage{displayRightFactor3}$

$\answer[format=string]{[} \answer[tolerance=0.05]{\sage{solution3[0][0]}}, \answer{\sage{Infinity}} \answer[format=string]{)}$

\begin{hint}
There are four boxes so you can input the entire interval. Each option should be: \\
( or [ \\
number or $\infty$ \\
number or $\infty$ \\
) or ] 
\end{hint}

\end{question}

\begin{question}
$\sage{displayLeftFactor4} \geq \sage{displayRightFactor4}$

$\answer[format=string]{(} \answer{\sage{-Infinity}}, \answer[tolerance=0.05]{\sage{solution4[0][1]}} \answer[format=string]{]}$

\end{question}

\begin{question}
$\sage{displayLeftFactor5} < \sage{displayRightFactor5}$

$\answer[format=string]{(} \answer{\sage{-Infinity}}, \answer[tolerance=0.05]{\sage{solution5[0][1]}} \answer[format=string]{)}$

\end{question}

\begin{question}
$\sage{displayLeftFactor6} > \sage{displayRightFactor6}$

$\answer[format=string]{(} \answer[tolerance=0.05]{\sage{solution6[0][0]}}, \answer{\sage{Infinity}} \answer[format=string]{)}$

\end{question}

\begin{sagesilent}
import random
x = var('x')
##################
def maybeMakeNegative(rational):
    maybeNegative = (-1)**ZZ.random_element(2)
    rational = maybeNegative * rational
    return rational
    
def createNumerators():
    numerators = [0, 0, 0, 0]
    numerators[0] = maybeMakeNegative(ZZ.random_element(3, 11))
    numerators[1] = maybeMakeNegative(ZZ.random_element(3, 11))
    numerators[2] = maybeMakeNegative(ZZ.random_element(3, 11))
    numerators[3] = maybeMakeNegative(ZZ.random_element(3, 11))
    return numerators

def createDenominators():
    listOfDenominators= range(2, 10)
    a, b, c, d = random.sample(listOfDenominators, 4)
    return [Integer(a), Integer(b), Integer(c), Integer(d)]

def createCoefficientsHard():
    n0, n1, n2, n3 = createNumerators()
    d0, d1, d2, d3 = createDenominators()
    left = (n0/d0)+(n1/d1)*x
    right = (n2/d2)*x+(n3/d3)
    oneSolutionCheck = (n1/d1) - (n2/d2)
    while oneSolutionCheck == 0:
        n0, n1, n2, n3 = createNumerators()
        d0, d1, d2, d3 = createDenominators()
        left = (n0/d0)+(n1/d1)*x
        right = (n2/d2)*x+(n3/d3)
        oneSolutionCheck = (n1/d1) - (n2/d2)
    return [n0, n1, n2, n3, d0, d1, d2, d3]

def createIntervalSolutionHard(coefficients, direction):
    n0, n1, n2, n3, d0, d1, d2, d3 = coefficients
    left = (n0/d0)+(n1/d1)*x
    right = (n2/d2)*x+(n3/d3)
    # Checks which direction the interval points
    endpoint = round(float(solve(left-right, x)[0].rhs() ), 3)
    #print "Check this info: \n Coefficients: %s \n Left: %s \n Right: %s \n Solution: %s" %(coefficients, left, right, endpoint)
    nearEndpoint = endpoint - 1
    checkNearby = float(float(n0/d0)+float(n1/d1)*nearEndpoint - (float(n2/d2)*nearEndpoint+float(n3/d3)))
    if direction == "leq" or direction == "less":
        if (checkNearby < 0):
            interval = [-oo, endpoint]
            whichSideFloat = "Right"
        else:
            interval = [endpoint, oo]
            whichSideFloat = "Left"  
    elif direction == "geq" or direction == "greater": 
        if (checkNearby > 0):
            interval = [-oo, endpoint]
            whichSideFloat = "Right"
        else:
            interval = [endpoint, oo]
            whichSideFloat = "Left"
    else:
        print "Input an invalid type of inequality"
        interval = [0, 0]
        whichSideFloat = "Neither"
    return [interval, whichSideFloat]

######## QUESTION 7 - Forces $(a, \infty)$ solution ##########
coefficients7 = createCoefficientsHard()
solution7 = createIntervalSolutionHard(coefficients7, "less")
while solution7[1] == "Right":
    coefficients7 = createCoefficientsHard()
    solution7 = createIntervalSolutionHard(coefficients7, "less")
n70, n71, n72, n73, d70, d71, d72, d73 = coefficients7
displayLeftFactor7 = n70/d70+n71/d71*x
displayRightFactor7 = n72/d72*x+n73/d73

######## QUESTION 8 - Forces $(a, \infty)$ solution ##########
coefficients8 = createCoefficientsHard()
solution8 = createIntervalSolutionHard(coefficients8, "greater")
while solution8[1] == "Right":
    coefficients8 = createCoefficientsHard()
    solution8 = createIntervalSolutionHard(coefficients8, "greater")
n80, n81, n82, n83, d80, d81, d82, d83 = coefficients8
displayLeftFactor8 = n80/d80+n81/d81*x
displayRightFactor8 = n82/d82*x+n83/d83

######## QUESTION 9 - Forces $(-\infty, a)$ solution ##########
coefficients9 = createCoefficientsHard()
solution9 = createIntervalSolutionHard(coefficients9, "leq")
while solution9[1] == "Left":
    coefficients9 = createCoefficientsHard()
    solution9 = createIntervalSolutionHard(coefficients9, "leq")
n90, n91, n92, n93, d90, d91, d92, d93 = coefficients9
displayLeftFactor9 = n90/d90+n91/d91*x
displayRightFactor9 = n92/d92*x+n93/d93

######## QUESTION 10 - Forces $(-\infty, a)$ solution ##########
coefficients10 = createCoefficientsHard()
solution10 = createIntervalSolutionHard(coefficients10, "geq")
while solution10[1] == "Left":
    coefficients10 = createCoefficientsHard()
    solution10 = createIntervalSolutionHard(coefficients10, "geq")
n100, n101, n102, n103, d100, d101, d102, d103 = coefficients10
displayLeftFactor10 = n100/d100+n101/d101*x
displayRightFactor10 = n102/d102*x+n103/d103

\end{sagesilent}

\begin{question}
$\sage{displayLeftFactor7} < \sage{displayRightFactor7}$

$\answer[format=string]{(} \answer[tolerance=0.05]{\sage{solution7[0][0]}}, \answer{\sage{Infinity}} \answer[format=string]{)}$

\end{question}

\begin{question}
$\sage{displayLeftFactor8} > \sage{displayRightFactor8}$

$\answer[format=string]{(} \answer[tolerance=0.05]{\sage{solution8[0][0]}}, \answer{\sage{Infinity}} \answer[format=string]{)}$

\end{question}

\begin{question}
$\sage{displayLeftFactor9} \leq \sage{displayRightFactor9}$

$\answer[format=string]{(} \answer{\sage{-Infinity}}, \answer[tolerance=0.05]{\sage{solution9[0][1]}} \answer[format=string]{]}$

\end{question}

\begin{question}
$\sage{displayLeftFactor10} \geq \sage{displayRightFactor10}$

$\answer[format=string]{(} \answer{\sage{-Infinity}}, \answer[tolerance=0.05]{\sage{solution10[0][1]}} \answer[format=string]{]}$

\end{question}

\end{document}
