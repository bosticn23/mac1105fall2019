\documentclass{ximera}
%\usepackage{sagetex}
%% handout
%% space
%% newpage
%% numbers
%% nooutcomes

%%% You can put user macros here
%% However, you cannot make new environments

\graphicspath{{./}{module1Activity/}{module2Activity/}{module3Activity/}}

\usepackage{sagetex}
\usepackage{tikz}
\usepackage{hyperref}
\usepackage{tkz-euclide}
\usetkzobj{all}
\pgfplotsset{compat=1.7} % prevents compile error.

\tikzstyle geometryDiagrams=[ultra thick,color=blue!50!black]
 %% we can turn off input when making a master document

\outcome{}
\author{Darryl Chamberlain Jr.}
 
\title{Objectives}

\begin{document}
\begin{abstract}
List of objectives for Module 10M - Modeling with Power Functions.
\end{abstract}
\maketitle

% Introduction to the section. 
In the last Module, we focused on modeling with linear functions. These models had a \textbf{constant} rate of change, which is how we identified the need for a linear model. In this Module, we'll focus on models that have a \textbf{varying} rate of change. We will use polynomials, radical, and rational functions to model these relationships. These are all known as \textit{Power Functions} as they describe a function using powers of $x$. 

\begin{comment}
Linear functions are power functions as well!
\end{comment}

The objectives for this Module are: 
\begin{enumerate}
    \item \href{https://cnx.org/contents/mwjClAV_@8.12:yUH0hROr@12/Modeling-Using-Variation}{Identify when two quantities are varying directly with each other.}
	\item \href{https://cnx.org/contents/mwjClAV_@8.12:yUH0hROr@12/Modeling-Using-Variation}{Identify when two quantities are varying indirectly with each other.}
	\item \href{https://cnx.org/contents/mwjClAV_@8.12:yUH0hROr@12/Modeling-Using-Variation}{Construct a model equation for the real-life situation.} 
\end{enumerate}

\end{document}