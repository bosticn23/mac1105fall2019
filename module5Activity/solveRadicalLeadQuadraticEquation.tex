\documentclass{ximera}
\usepackage{sagetex}
%% handout
%% space
%% newpage
%% numbers
%% nooutcomes

%% You can put user macros here
%% However, you cannot make new environments

\graphicspath{{./}{module1Activity/}{module2Activity/}{module3Activity/}}

\usepackage{sagetex}
\usepackage{tikz}
\usepackage{hyperref}
\usepackage{tkz-euclide}
\usetkzobj{all}
\pgfplotsset{compat=1.7} % prevents compile error.

\tikzstyle geometryDiagrams=[ultra thick,color=blue!50!black]
 %% we can turn off input when making a master document

\outcome{}
\author{Darryl Chamberlain Jr.}
 
\title{Objective 4 - Solving Radical Equations (Quadratic)}

\begin{document}
\begin{abstract}
Solve radical equations that lead to quadratic equations. 
\end{abstract}
\maketitle

\href{https://cnx.org/contents/mwjClAV_@8.1:uI1As6DV@15/Other-Types-of-Equations}{Link to section in online textbook.}

First, watch 
% UPDATE VIDEO LINK
\underline{\href{https://mediasite.video.ufl.edu/Mediasite/Play/631ca9eb67f349698705032a7c42906d1d}{this video}} to see how solving radical equations is different from solving linear and quadratic equations. \textbf{The major difference is in the restricted domains of radical functions!} This objective will focus on radical equations that lead to quadratic equations. That means we can have 0, 1, or 2 solutions \textit{(based on whether the potential solutions are in the domains of the radical functions)}. 

%%%%%%%%%%%%%%%%%%%%%
%%%  Objective 4  %%%
%%%%%%%%%%%%%%%%%%%%%


%Now practice working with 
% TYPE OF PROBLEMS
%below. 
% (ax+b)(cx+d) = acx^2 + (ad+bc)x + bd  
% (ac)x^2 + bd = - (ad+bc)x
% \sqrt{(ac)x^2 + bd} = \sqrt{- (ad+bc)x} 
%  \sqrt{(ac)x^2 + bd} - \sqrt{- (ad+bc)x} = 0

\begin{sagesilent}
x = var('x')
def generateCoefficients(): 
    a = (-1)**ZZ.random_element(2) * (ZZ.random_element(3, 7))
    b = (-1)**ZZ.random_element(2) * (ZZ.random_element(3, 7))
    c = (-1)**ZZ.random_element(2) * (ZZ.random_element(3, 7))
    d = (-1)**ZZ.random_element(2) * (ZZ.random_element(3, 7))
    discriminant = (a*c+b*d)**2 - 4*a*c*b*d
    while discriminant < 0 or discriminant == 0:
        a = (-1)**ZZ.random_element(2) * (ZZ.random_element(3, 7))
        b = (-1)**ZZ.random_element(2) * (ZZ.random_element(3, 7))
        c = (-1)**ZZ.random_element(2) * (ZZ.random_element(3, 7))
        d = (-1)**ZZ.random_element(2) * (ZZ.random_element(3, 7))
        discriminant = (a*c+b*d)**2 - 4*a*c*b*d
    return [a, b, c, d]
#  
def checkDomainOfTerms(a, b, c, d, solution):
    #print "Coefficients: %d, %d, %d, %d" %(a, b, c, d)
    #print "Solution: %s" %solution
    term1 = a*c*x**2 + b*d
    term2 = -(a*d+b*c)*x
    term1Check = term1(x=solution)
    term2Check = term2(x=solution)
    print "Check 1: %s and Check 2: %s" %(term1Check, term2Check)
    if term1Check > 0 and term2Check > 0:
        inDomain = "True"
    else:
        inDomain = "False"
    return inDomain
#
def equationWithTwoSolutions():
    a, b, c, d = generateCoefficients()
    solution1 = -b/a
    solution2 = -d/c
    solution1inDomain = checkDomainOfTerms(a, b, c, d, solution1)
    solution2inDomain = checkDomainOfTerms(a, b, c, d, solution2)
    counter = 0
    while solution1inDomain == "False" or solution2inDomain == "False":
        a, b, c, d = generateCoefficients()
        solution1 = -b/a
        solution2 = -d/c
        solution1inDomain = checkDomainOfTerms(a, b, c, d, solution1)
        solution2inDomain = checkDomainOfTerms(a, b, c, d, solution2)
        counter = counter + 1 
        print "Counter: %d" %counter
    return [a, b, c, d]
#
def equationWithOneSolution():
    a, b, c, d = generateCoefficients()
    solution1 = -b/a
    solution2 = -d/c
    solution1inDomain = checkDomainOfTerms(a, b, c, d, solution1)
    solution2inDomain = checkDomainOfTerms(a, b, c, d, solution2)
    counter = 0
    while (solution1inDomain == "False" and solution2inDomain == "False") or (solution1inDomain == "True" and solution2inDomain == "True"):
        a, b, c, d = generateCoefficients()
        solution1 = -b/a
        solution2 = -d/c
        solution1inDomain = checkDomainOfTerms(a, b, c, d, solution1)
        solution2inDomain = checkDomainOfTerms(a, b, c, d, solution2)
        counter = counter + 1 
        print "Counter: %d" %counter
    if solution1inDomain == "True":
        return [a, b, c, d, round(solution1, 3)]
    else:
        return [a, b, c, d, round(solution2, 3)]
#
def equationWithZeroSolutions():
    a, b, c, d = generateCoefficients()
    solution1 = -b/a
    solution2 = -d/c
    solution1inDomain = checkDomainOfTerms(a, b, c, d, solution1)
    solution2inDomain = checkDomainOfTerms(a, b, c, d, solution2)
    counter = 0
    while solution1inDomain == "True" or solution2inDomain == "True":
        a, b, c, d = generateCoefficients()
        solution1 = -b/a
        solution2 = -d/c
        solution1inDomain = checkDomainOfTerms(a, b, c, d, solution1)
        solution2inDomain = checkDomainOfTerms(a, b, c, d, solution2)
        counter = counter + 1 
        print "Counter: %d" %counter
    return [a, b, c, d]
#
def orderSolutions(a, b, c, d):
    s1 = round(-b/a, 3)
    s2 = round(-d/c, 3)
    if (s1 < s2):
        return [s1, s2]
    else: 
        return [s2, s1]
##### END OF DEFINITIONS #####

##### QUESTION 13 #####
a13, b13, c13, d13 = equationWithTwoSolutions()
termA13 = a13*c13*x**2 + b13*d13
termB13 = -(a13*d13+b13*c13)*x
solutionA13, solutionB13 = orderSolutions(a13, b13, c13, d13)

##### QUESTION 14 #####
a14, b14, c14, d14, solution14 = equationWithOneSolution()
termA14 = a14*c14*x**2 + b14*d14
termB14 = -(a14*d14+b14*c14)*x

##### QUESTION 15 #####
a15, b15, c15, d15 = equationWithZeroSolutions()
termA15 = a15*c15*x**2 + b15*d15
termB15 = -(a15*d15+b15*c15)*x
\end{sagesilent}

%Q 13 
\begin{question}
Solve the following equation. 

$$ \sqrt{\sage{termA13}} - \sqrt{\sage{termB13}} = 0 $$

Smallest solution: $x = \answer[tolerance=0.05]{\sage{solutionA13}}$

Largest solution: $x = \answer[tolerance=0.05]{\sage{solutionB13}}$

\textit{If there is only one Real solution, type "NA" as the largest solution. If there are no Real solutions, type "NA" for both.}

\end{question}

%Q 14
\begin{question}
Solve the following equation. 

$$ \sqrt{\sage{termA14}} - \sqrt{\sage{termB14}} = 0 $$

Smallest solution: $x = \answer[tolerance=0.05]{\sage{solution14}}$

Largest solution: $x = \answer[format=string]{NA}$

\textit{If there is only one Real solution, type "NA" as the largest solution. If there are no Real solutions, type "NA" for both.}
\end{question}

%Q 15
\begin{question}
Solve the following equation. 

$$ \sqrt{\sage{termA15}} - \sqrt{\sage{termB15}} = 0 $$

Smallest solution: $x = \answer[format=string]{NA}$

Largest solution: $x = \answer[format=string]{NA}$

\textit{If there is only one Real solution, type "NA" as the largest solution. If there are no Real solutions, type "NA" for both.}
\end{question}

\end{document}
