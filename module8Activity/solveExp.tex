\documentclass{ximera}
\usepackage{sagetex}
%% handout
%% space
%% newpage
%% numbers
%% nooutcomes

%% You can put user macros here
%% However, you cannot make new environments

\graphicspath{{./}{module1Activity/}{module2Activity/}{module3Activity/}}

\usepackage{sagetex}
\usepackage{tikz}
\usepackage{hyperref}
\usepackage{tkz-euclide}
\usetkzobj{all}
\pgfplotsset{compat=1.7} % prevents compile error.

\tikzstyle geometryDiagrams=[ultra thick,color=blue!50!black]
 %% we can turn off input when making a master document

\outcome{}
\author{Darryl Chamberlain Jr.}
 
\title{Objective 4 - Solve Exponential}

\begin{document}
\begin{abstract}
Solve exponential equations with the same or different bases.
\end{abstract}
\maketitle

\href{https://cnx.org/contents/mwjClAV_@8.1:wfhkHyEh@11/Exponential-and-Logarithmic-Equations}{Link to section in online textbook.}

%%%%%%%%%%%%%%%%%%%%%
%%%  Objective 2  %%%
%%%%%%%%%%%%%%%%%%%%%

First, watch 
% UPDATE VIDEO LINK
\underline{\href{https://mediasite.video.ufl.edu/Mediasite/Play/a70362f0ebae4b12a37022c13921532c1d}{this video}} to learn how to solve certain types of exponential equations (when the bases are the same). Then practice this method below.

\begin{sagesilent}
x = var("x")

def maybeMakeNegative(natural):
    integer = natural*(-1)**ZZ.random_element(2)
    return integer 

def generateFactor():
    a0 = maybeMakeNegative(ZZ.random_element(2, 7))
    a1 = maybeMakeNegative(ZZ.random_element(2, 7))
    factor = a0*x + a1
    return factor 

#####
factor5a = generateFactor()
factor5b = generateFactor()
base5 = ZZ.random_element(2, 7)
#
while len(solve(factor5a - factor5b == 0, x)) == 0:
    factor5a = generateFactor()
    factor5b = generateFactor()
    base5 = ZZ.random_element(2, 7)
#
solution5 = round(float( solve(factor5a - factor5b == 0, x)[0].rhs() ), 3)
#####
#####
factor6a = generateFactor()
factor6b = generateFactor()
preBase6 = ZZ.random_element(2, 7)
k6 = ZZ.random_element(2, 4)
base6a = preBase6**k6
base6b = preBase6
#
while len(solve(k6*factor6a - factor6b == 0, x)) == 0:
    factor6a = generateFactor()
    factor6b = generateFactor()
    preBase6 = ZZ.random_element(2, 7)
    k6 = ZZ.random_element(2, 4)
    base6a = preBase6**k6
    base6b = preBase6
#
solution6 = round(float( solve(k6*factor6a - factor6b == 0, x)[0].rhs() ), 3)
#####
#####
factor7a = generateFactor()
factor7b = generateFactor()
preBase7 = ZZ.random_element(2, 7)
k7 = -ZZ.random_element(2, 4)
base7a = preBase7**k7
base7b = preBase7
#
while len(solve(k7*factor7a - factor7b == 0, x)) == 0:
    factor7a = generateFactor()
    factor7b = generateFactor()
    preBase7 = ZZ.random_element(2, 7)
    k7 = -ZZ.random_element(2, 4)
    base7a = preBase7**k7
    base7b = preBase7
#
solution7 = round(float( solve(k7*factor7a - factor7b == 0, x)[0].rhs() ), 3)
#####
#####
factor8a = generateFactor()
factor8b = generateFactor()
preBase8 = ZZ.random_element(2, 7)
k8a = ZZ.random_element(2, 4)
k8b = k8a + ZZ.random_element(1, 2)
base8a = preBase8**k8a
base8b = preBase8**k8b
#
while len(solve(k8a*factor8a - k8b*factor8b == 0, x)) == 0:
    factor8a = generateFactor()
    factor8b = generateFactor()
    preBase8 = ZZ.random_element(2, 7)
    k8a = ZZ.random_element(2, 4)
    k8b = k8a + ZZ.random_element(1, 2)
    base8a = preBase8**k8a
    base8b = preBase8**k8b
#
solution8 = round(float( solve(k8a*factor8a - k8b*factor8b == 0, x)[0].rhs() ), 3)
\end{sagesilent}

% Q5
\begin{question}
Solve the exponential equation below.

$$ \sage{base5}^{\sage{factor5a}} = \sage{base5}^{\sage{factor5b}} $$

$ x = \answer[tolerance=0.05]{\sage{solution5}} $
\end{question}

% Q6
\begin{question}
Solve the exponential equation below. 

$$ \sage{base6a}^{\sage{factor6a}} = \sage{base6b}^{\sage{factor6b}} $$

$ x = \answer[tolerance=0.05]{\sage{solution6}} $
\end{question}

% Q7
\begin{question}
Solve the exponential equation below. 

$$ \left(\sage{base7a}\right)^{\sage{factor7a}} = \sage{base7b}^{\sage{factor7b}} $$

$ x = \answer[tolerance=0.05]{\sage{solution7}} $
\end{question}

% Q8
\begin{question}
Solve the exponential equation below. 

$$ \sage{base8a}^{\sage{factor8a}} = \sage{base8b}^{\sage{factor8b}} $$

$ x = \answer[tolerance=0.05]{\sage{solution8}} $
\end{question}

It's great when equations line up like this, but what happens if we \textbf{can't} write the bases to be the same? Watch \underline{\href{https://mediasite.video.ufl.edu/Mediasite/Play/74d3379663e044208baceb58ec14428b1d}{this video}} to learn how to solve \textbf{any} type of exponential equation. Then practice with the questions below. \\ 

\textit{Note: Yes, you can use this method all of the time! While changing the bases to be the same to rewrite the question can be easier sometimes, taking the log of both sides will \textbf{always} solve exponential equations.}

\begin{sagesilent}
x = var("x")

def maybeMakeNegative(natural):
    integer = natural*(-1)**ZZ.random_element(2)
    return integer 

def generateFactor():
    a0 = maybeMakeNegative(ZZ.random_element(2, 7))
    a1 = maybeMakeNegative(ZZ.random_element(2, 7))
    factor = a0*x + a1
    return factor 

def generateRelativelyPrimeBases():
    b0 = ZZ.random_element(2, 9)
    b1 = ZZ.random_element(2, 9)
    while gcd(b0, b1) > 1:
        b0 = ZZ.random_element(2, 9)
        b1 = ZZ.random_element(2, 9)
    return [b0, b1]
##########
factor9a = generateFactor()
factor9b = generateFactor()
base9a, base9b = generateRelativelyPrimeBases()
#
while len(solve(factor9a * ln(base9a) - factor9b * ln(base9b) == 0, x)) == 0: 
    factor9a = generateFactor()
    factor9b = generateFactor()
    base9a, base9b = generateRelativelyPrimeBases()
#
solution9 = round(float( solve(factor9a * ln(base9a) - factor9b * ln(base9b) == 0, x)[0].rhs() ), 3)
#####
#####
factor10a = generateFactor()
factor10b = generateFactor()
base10a = generateRelativelyPrimeBases()[0] ** ZZ.random_element(2, 4)
base10b = generateRelativelyPrimeBases()[1] ** ZZ.random_element(2, 4)
#
while len(solve(factor10a * ln(base10a) - factor10b * ln(base10b) == 0, x)) == 0:
    factor10a = generateFactor()
    factor10b = generateFactor()
    base10a = generateRelativelyPrimeBases()[0] ** ZZ.random_element(2, 4)
    base10b = generateRelativelyPrimeBases()[1] ** ZZ.random_element(2, 4)
#
solution10 = round(float( solve(factor10a * ln(base10a) - factor10b * ln(base10b) == 0, x)[0].rhs() ), 3)
#####
#####
factor11a = generateFactor()
factor11b = generateFactor()
base11a = generateRelativelyPrimeBases()[0] ** ZZ.random_element(2, 4)
base11b = generateRelativelyPrimeBases()[1] ** - ZZ.random_element(2, 4)
#
while len(solve(factor11a * ln(base11a) - factor11b * ln(base11b) == 0, x)) == 0:
    factor11a = generateFactor()
    factor11b = generateFactor()
    base11a = generateRelativelyPrimeBases()[0] ** ZZ.random_element(2, 4)
    base11b = generateRelativelyPrimeBases()[1] ** - ZZ.random_element(2, 4)
#
solution11 = round(float( solve(factor11a * ln(base11a) - factor11b * ln(base11b) == 0, x)[0].rhs() ), 3)
\end{sagesilent}

% Q9
\begin{question}
Solve the exponential equation below. 

$$ \sage{base9a}^{\sage{factor9a}} = \sage{base9b}^{\sage{factor9b}} $$

$ x = \answer[tolerance=0.05]{\sage{solution9}} $
\end{question}

% Q10
\begin{question}
Solve the exponential equation below. 

$$ \sage{base10a}^{\sage{factor10a}} = \sage{base10b}^{\sage{factor10b}} $$

$ x = \answer[tolerance=0.05]{\sage{solution10}} $
\end{question}

% Q11
\begin{question}
Solve the exponential equation below. 

$$ \sage{base11a}^{\sage{factor11a}} = \sage{base11b}^{\sage{factor11b}} $$

$ x = \answer[tolerance=0.05]{\sage{solution11}} $
\end{question}

\begin{question}
\textbf{Main takeaway:} Before looking, you should work through the previous problems. \textit{Have you finished working through the examples?} $\answer[format=string]{Yes}$
\begin{feedback}[correct]
While changing the bases to be the same to rewrite the question can be easier sometimes, taking the log of both sides will \textbf{always} solve exponential equations.
\end{feedback}
\end{question}

\end{document}
