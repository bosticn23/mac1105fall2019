\documentclass{ximera}
\usepackage{sagetex}
\usepackage{multicol}
%% handout
%% space
%% newpage
%% numbers
%% nooutcomes

%% You can put user macros here
%% However, you cannot make new environments

\graphicspath{{./}{module1Activity/}{module2Activity/}{module3Activity/}}

\usepackage{sagetex}
\usepackage{tikz}
\usepackage{hyperref}
\usepackage{tkz-euclide}
\usetkzobj{all}
\pgfplotsset{compat=1.7} % prevents compile error.

\tikzstyle geometryDiagrams=[ultra thick,color=blue!50!black]
 %% we can turn off input when making a master document

\outcome{Understand and solve quadratic equations.}
\author{Darryl Chamberlain Jr.}
 
\title{Objective 3 - Solving Quadratics by Factoring}

\begin{document}
\begin{abstract}
Solving quadratic equations using the zero-factor principle.
\end{abstract}
\maketitle

\href{https://cnx.org/contents/mwjClAV_@8.1:-Sm9he1Q@17/Quadratic-Functions}{Link to section in online textbook.}

%%%%%%%%%%%%%%%%%%%%%
%%%  Objective 2  %%%
%%%%%%%%%%%%%%%%%%%%%

First, watch \underline{\href{https://mediasite.video.ufl.edu/Mediasite/Play/3b62004c61964849bdaa95a9cec047531d}{this video}} to learn how to solve quadratic equations through factoring. If you are having trouble factoring, look back at Module 2. Feel free to pause the video and fill out the notes as you go. 

Now practice solving quadratic equations by factoring. 

\begin{sagesilent}
x = var("x")

###########################
def maybeMakeNegative(natural):
    integer = natural*(-1)**ZZ.random_element(2)
    return integer

def generateFactors(minimumPrime, maximumPrime, numberOfFactors):
    listPrimes = [p for p in range(minimumPrime, maximumPrime+1) if is_prime(p)]
    print "List of Primes: %s" %listPrimes
    aFactors = [listPrimes[ZZ.random_element(0, len(listPrimes))] for i in range(numberOfFactors)]
    print "a Factors: %s" %aFactors
    cFactors = [listPrimes[ZZ.random_element(0, len(listPrimes))] for i in range(numberOfFactors)]
    print "c Factors: %s" %cFactors
    return [aFactors, cFactors]
    
def generateSolution(minimum, maximum, numberOfFactors):
    factors = generateFactors(minimum, maximum, numberOfFactors)
    aFactors = factors[0]
    cFactors = factors[1]
    a = prod(aFactors)
    c = prod(cFactors)
    b = maybeMakeNegative(ZZ.random_element(minimum, maximum+1))
    d = maybeMakeNegative(ZZ.random_element(minimum, maximum+1))
    #This will guarantee that we always generate solutions with b < d
    if(b < d):
        return [a, b, c, d]
    else:
        return [c, d, a, b]

def generateZeros(a, b, c, d):    
    z0 = round(float(-b/a), 3)
    z1 = round(float(-d/c), 3)
    # Orders from small to large
    if z0 < z1:
        return [z0, z1]
    else:
        return [z1, z0]

def generateProblem(solution):
    a, b, c, d = solution
    return [a*c, a*d + b*c, b*d]

############ END OF DEFINITIONS ###############
numberOfFactors567 = 1
minimum = 2
maximum = 7

##### QUESTION 5 #####
solution5 = generateSolution(minimum, maximum, numberOfFactors567)
problem5 = generateProblem(solution5)
generalForm5 = problem5[0] * x**2 + problem5[1] * x + problem5[2]
zeros5 = generateZeros(solution5[0], solution5[1], solution5[2], solution5[3])

##### QUESTION 6 #####
solution6 = generateSolution(minimum, maximum, numberOfFactors567)
problem6 = generateProblem(solution6)
generalForm6 = problem6[0] * x**2 + problem6[1] * x + problem6[2]
zeros6 = generateZeros(solution6[0], solution6[1], solution6[2], solution6[3])

##### QUESTION 7 #####
solution7 = generateSolution(minimum, maximum, numberOfFactors567)
problem7 = generateProblem(solution7)
generalForm7 = problem7[0] * x**2 + problem7[1] * x + problem7[2]
zeros7 = generateZeros(solution7[0], solution7[1], solution7[2], solution7[3])

####
numberOfFactors8910 = 2

##### Question 8 #####
solution8 = generateSolution(minimum, maximum, numberOfFactors8910)
problem8 = generateProblem(solution8)
generalForm8 = problem8[0] * x**2 + problem8[1] * x + problem8[2]
zeros8 = generateZeros(solution8[0], solution8[1], solution8[2], solution8[3])

##### Question 9 #####
solution9 = generateSolution(minimum, maximum, numberOfFactors8910)
problem9 = generateProblem(solution9)
generalForm9 = problem9[0] * x**2 + problem9[1] * x + problem9[2]
zeros9 = generateZeros(solution9[0], solution9[1], solution9[2], solution9[3])

##### Question 10 #####
solution10 = generateSolution(minimum, maximum, numberOfFactors8910)
problem10 = generateProblem(solution10)
generalForm10 = problem10[0] * x**2 + problem10[1] * x + problem10[2]
zeros10 = generateZeros(solution10[0], solution10[1], solution10[2], solution10[3])
\end{sagesilent}

\begin{question}
Solve the quadratic equation below. 

	$$ \sage{generalForm5} = 0 $$

Smaller solution: $x = \answer[tolerance=0.05]{\sage{zeros5[0]}}$

Larger solution: $x = \answer[tolerance=0.05]{\sage{zeros5[1]}}$ 

\end{question}

\begin{question}
Solve the quadratic equation below. 

	$$ \sage{generalForm6} = 0 $$
	
Smaller solution: $x = \answer[tolerance=0.05]{\sage{zeros6[0]}}$

Larger solution: $x = \answer[tolerance=0.05]{\sage{zeros6[1]}}$ 

\end{question}

\begin{question}
Solve the quadratic equation below. 

	$$ \sage{generalForm7} = 0 $$
	
Smaller solution: $x = \answer[tolerance=0.05]{\sage{zeros7[0]}}$

Larger solution: $x = \answer[tolerance=0.05]{\sage{zeros7[1]}}$ 

\end{question}

%TYPE 2 (Type 3 from M2O4) - a and c are fairly composite (ac has at least 8 factors)
\begin{question}
Solve the quadratic equation below. 

	$$ \sage{generalForm8} = 0 $$
	
Smaller solution: $x = \answer[tolerance=0.05]{\sage{zeros8[0]}}$

Larger solution: $x = \answer[tolerance=0.05]{\sage{zeros8[1]}}$ 

\end{question}

\begin{question}
Solve the quadratic equation below. 

	$$ \sage{generalForm9} = 0 $$
	
Smaller solution: $x = \answer[tolerance=0.05]{\sage{zeros9[0]}}$

Larger solution: $x = \answer[tolerance=0.05]{\sage{zeros9[1]}}$ 

\end{question}

\begin{question}
Solve the quadratic equation below. 

	$$ \sage{generalForm10} = 0 $$
	
Smaller solution: $x = \answer[tolerance=0.05]{\sage{zeros10[0]}}$

Larger solution: $x = \answer[tolerance=0.05]{\sage{zeros10[1]}}$ 

\end{question}

\end{document}
