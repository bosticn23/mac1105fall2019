\documentclass{ximera}
%\usepackage{sagetex}
%% handout
%% space
%% newpage
%% numbers
%% nooutcomes

%%% You can put user macros here
%% However, you cannot make new environments

\graphicspath{{./}{module1Activity/}{module2Activity/}{module3Activity/}}

\usepackage{sagetex}
\usepackage{tikz}
\usepackage{hyperref}
\usepackage{tkz-euclide}
\usetkzobj{all}
\pgfplotsset{compat=1.7} % prevents compile error.

\tikzstyle geometryDiagrams=[ultra thick,color=blue!50!black]
 %% we can turn off input when making a master document

\outcome{}
\author{Darryl Chamberlain Jr.}
 
\title{Objectives}

\begin{document}
\begin{abstract}
List of objectives for Module 11L - Introduction to Limits.
\end{abstract}
\maketitle

% Introduction to the section. 
College Algebra textbooks normally avoid the formal notation of the limit and leave it for the Calculus textbooks. But this is something we have talked about already when graphing polynomials and rational functions! With it, we can describe the behavior of many more functions and allow us to get a better understanding of our elementary functions: polynomials, rational, radical, logarithmic, and exponential functions. 

The objectives for this homework are: 
\begin{enumerate}
\item  \href{https://cnx.org/contents/i4nRcikn@5.1:dKCfyV9u@5/The-Limit-of-a-Function}{Interpret the notation for limits.}
\item \href{https://cnx.org/contents/i4nRcikn@5.1:dKCfyV9u@5/The-Limit-of-a-Function}{Evaluate the left or right limit of a function.}
\item \href{https://cnx.org/contents/i4nRcikn@5.1:dKCfyV9u@5/The-Limit-of-a-Function}{Evaluate the limit of a function.}
\end{enumerate}

\textit{Note: Yes this is using the Calculus I textbook! We are going to move through the material slowly and in the context of the functions we have seen so far in this course. We are going to learn just enough to learn how to graph more complicated rational functions (which will be Module 12).}

\end{document}