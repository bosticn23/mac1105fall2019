\documentclass{ximera}
\usepackage{sagetex}
%% handout
%% space
%% newpage
%% numbers
%% nooutcomes

%% You can put user macros here
%% However, you cannot make new environments

\graphicspath{{./}{module1Activity/}{module2Activity/}{module3Activity/}}

\usepackage{sagetex}
\usepackage{tikz}
\usepackage{hyperref}
\usepackage{tkz-euclide}
\usetkzobj{all}
\pgfplotsset{compat=1.7} % prevents compile error.

\tikzstyle geometryDiagrams=[ultra thick,color=blue!50!black]
 %% we can turn off input when making a master document

\outcome{Recognize and construct linear functions as well as solve linear equations.}
\author{Darryl Chamberlain Jr.}
 
\title{Objective 2 - Converting between linear forms}

\begin{document}
\begin{abstract}
Converting between Slope-Intercept form and Standard form.
\end{abstract}
\maketitle

\href{https://cnx.org/contents/mwjClAV_@8.1:62_eXnY6@14/Linear-Equations-in-One-Variable}{Link to section in online textbook.}

%%%%%%%%%%%%%%%%%%%%%
%%%  Objective 1  %%%
%%%%%%%%%%%%%%%%%%%%%

First, watch \underline{\href{https://mediasite.video.ufl.edu/Mediasite/Play/49d31e3bbea64e158f4481960444eb791d}{this video}} to learn about the different forms we usually write linear functions in. This objective will focus on converting between Slope-Intercept form and Standard form.

\begin{sagesilent}
x = var('x')
y = var('y')

#################
def maybeMakeNegative(rational):
    maybeNegative = (-1)**ZZ.random_element(2)
    rational = maybeNegative * rational
    return rational

def standardToSlopeInt():
    A = ZZ.random_element(2, 9)
    B = maybeMakeNegative(ZZ.random_element(2, 9))
    C = maybeMakeNegative(ZZ.random_element(2, 9))
    while gcd(A, B) > 1 or gcd(A, C) > 1:
        A = ZZ.random_element(2, 9)
        B = maybeMakeNegative(ZZ.random_element(2, 9))
        C = maybeMakeNegative(ZZ.random_element(2, 9))
    slope = float(-A/B)
    yInt = float(C/B)
    return [A, B, C, slope, yInt]

def sloptIntToStandard():
    mNum = maybeMakeNegative(ZZ.random_element(2, 9))
    mDenom = ZZ.random_element(2, 9)
    yIntNum = maybeMakeNegative(ZZ.random_element(2, 9))
    yIntDenom = ZZ.random_element(2, 9)
    while gcd(mNum, mDenom) > 1 or gcd(yIntNum, yIntDenom) > 1 or mDenom == yIntDenom:
        mNum = maybeMakeNegative(ZZ.random_element(2, 9))
        mDenom = ZZ.random_element(2, 9)
        yIntNum = maybeMakeNegative(ZZ.random_element(2, 9))
        yIntDenom = ZZ.random_element(2, 9)
    slope = mNum/mDenom
    yInt = yIntNum/yIntDenom
    lcmConstant = (mDenom*yIntDenom/gcd(mDenom, yIntDenom))
    if mNum < 0:
        A = -mNum*lcmConstant/mDenom
        B = lcmConstant
        C = yIntNum*lcmConstant/yIntDenom
    else:
        A = mNum*lcmConstant/mDenom
        B = -lcmConstant
        C = -yIntNum*lcmConstant/yIntDenom
    return [A, B, C, slope, yInt]
###############
A1, B1, C1, slope1, yInt1 = standardToSlopeInt()
equationXY1 = A1*x + B1*y
A2, B2, C2, slope2, yInt2 = standardToSlopeInt()
equationXY2 = A2*x + B2*y
A3, B3, C3, slope3, yInt3 = sloptIntToStandard()
equationSlopeInt3 = slope3*x + yInt3
A4, B4, C4, slope4, yInt4 = sloptIntToStandard()
equationSlopeInt4 = slope4*x + yInt4
\end{sagesilent}

% Question 1 - Standard to Slope-Intercept
\begin{question}
Convert the linear function below from Standard form to Slope-Intercept form. 

$$  \sage{equationXY1} = \sage{C1}  $$

$y = \answer[tolerance=0.5]{\sage{slope1}}x + \answer[tolerance=0.05]{\sage{yInt1}}$
\end{question}

% Question 2 - Standard to Slope-Intercept
\begin{question}
Convert the linear function below from Standard form to Slope-Intercept form. 

$$  \sage{equationXY2} = \sage{C2}  $$

$y = \answer[tolerance=0.5]{\sage{slope2}}x + \answer[tolerance=0.5]{\sage{yInt2}}$
\end{question}

% Question 3 - Slope-Intercept to Standard
\begin{question}
Convert the linear function below from Slope-Intercept form to Standard form. 

$$ y = \sage{equationSlopeInt3}$$

\begin{hint}
What do we know about the coefficients in Standard Form? Is there anything special about the coefficient for $x$?
\end{hint}

$ \answer{\sage{A3}}x + \answer{\sage{B3}}y = \answer{\sage{C3}} $
\end{question}

% Question 4 - Slope-Intercept to Standard
\begin{question}
Convert the linear function below from Slope-Intercept form to Standard form. 

$$ y = \sage{equationSlopeInt4}$$

\begin{hint}
What do we know about the coefficients in Standard Form? Is there anything special about the coefficient for $x$?
\end{hint}

$ \answer{\sage{A4}}x + \answer{\sage{B4}}y = \answer{\sage{C4}} $
\end{question}

\end{document}