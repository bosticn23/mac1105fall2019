\documentclass{ximera}
\usepackage{sagetex}
%% handout
%% space
%% newpage
%% numbers
%% nooutcomes

\input{../preamble} %% we can turn off input when making a master document

\outcome{Recognize and construct linear functions as well as solve linear equations.}
\author{Darryl Chamberlain Jr.}
 
\title{Objective 4 - Solve Linear Equations}

\begin{document}
\begin{abstract}
Solve linear equations. 
\end{abstract}
\maketitle

\href{https://cnx.org/contents/mwjClAV_@8.1:62_eXnY6@14/Linear-Equations-in-One-Variable}{Link to section in online textbook.}

%%%%%%%%%%%%%%%%%%%%%
%%%  Objective 2  %%%
%%%%%%%%%%%%%%%%%%%%%

Now, watch \underline{\href{https://mediasite.video.ufl.edu/Mediasite/Play/486c7ecd0ea14369bfc405492ae942f51d}{this video}} to review how to solve linear equations. These techniques will be used throughout most of the semester. Be sure to write notes to yourself that you can review later!

Now try to solve the following linear equations. 

\begin{sagesilent}
# SOLVES BASIC LINEAR EQUATIONS OF THE FORM
    # b0 * (b1 + b2 * x) = b3 ( x * b4 - b5)

x = var('x')

###################
def maybeMakeNegative(rational):
    maybeNegative = (-1)**ZZ.random_element(2)
    rational = maybeNegative * rational
    return rational

def generateBlocks():
    listNaturals = range(2, 16)
    n0, n1, n2, n3, n4, n5 = sample(listNaturals, 6)
    b0 = Integer(-n0)
    b1 = Integer(maybeMakeNegative(n1))
    b2 = Integer(maybeMakeNegative(n2))
    b3 = Integer(-n3)
    b4 = Integer(maybeMakeNegative(n4))
    b5 = Integer(maybeMakeNegative(n5))
    # Begin checking for one solution
    OneSolutionCheck = b0*b2 - b3*b4
    # Makes sure there is exactly one solution
    while (OneSolutionCheck == 0):
        listNaturals = range(2, 16)
        n0, n1, n2, n3, n4, n5 = sample(listNaturals, 6)
        b0 = Integer(-n0)
        b1 = Integer(maybeMakeNegative(n1))
        b2 = Integer(maybeMakeNegative(n2))
        b3 = Integer(-n3)
        b4 = Integer(maybeMakeNegative(n4))
        b5 = Integer(maybeMakeNegative(n5))
        # Begin checking for one solution
        OneSolutionCheck = b0*b2 - b3*b4
    blocks = [b0, b1, b2, b3, b4, b5]
    return blocks

#b0 * (b1 + b2 * x) =  b3 ( x * b4 - b5)
def generateSolution(blocks):
    a, b, c, d, e, f = blocks
    basicLinearEquation = a * (b + c * x) - d * ( x * e - f)
    solution = float(solve(basicLinearEquation, x)[0].rhs())
    return solution

######### END OF DEFINITIONS ##########

##### QUESTION 10 #####
blocks10 = generateBlocks()
answer10 = generateSolution(blocks10)
displayEquationLeft10 = blocks10[1]+blocks10[2]*x
displayEquationRight10 = blocks10[4]*x-blocks10[5]

##### QUESTION 11 #####
blocks11 = generateBlocks()
answer11 = generateSolution(blocks11)
displayEquationLeft11 = blocks11[1]+blocks11[2]*x
displayEquationRight11 = blocks11[4]*x-blocks11[5]

##### QUESTION 12 #####
blocks12 = generateBlocks()
answer12 = generateSolution(blocks12)
displayEquationLeft12 = blocks12[1]+blocks12[2]*x
displayEquationRight12 = blocks12[4]*x-blocks12[5]

\end{sagesilent}

\begin{question}
Solve the equation below. 

$\sage{blocks10[0]} (\sage{displayEquationLeft10}) = \sage{blocks10[3]} (\sage{displayEquationRight10})$

$x = \answer[tolerance=0.05]{\sage{answer10}}$
\end{question}

\begin{question}
Solve the equation below. 

$\sage{blocks11[0]} (\sage{displayEquationLeft11}) = \sage{blocks11[3]} (\sage{displayEquationRight11})$

$x = \answer[tolerance=0.05]{\sage{answer11}}$
\end{question}

\begin{question}
Solve the equation below. 

$\sage{blocks12[0]} (\sage{displayEquationLeft12}) = \sage{blocks12[3]} (\sage{displayEquationRight12})$

$x = \answer[tolerance=0.05]{\sage{answer12}}$
\end{question}

%%%%%%%%%% TYPE 2 - ADVANCED LINEAR EQUATIONS

\begin{sagesilent}
# Type 2 - Solve Advanced linear equations (fractions)
#(coefficients[0]*x + numerators[0])/denominators[0]
    # - (coefficients[1]*x+numerators[1])/denominators[1]
    # = (coefficients[2]*x+numerators[2])/denominators[2]

x = var('x')
###################
def maybeMakeNegative(rational):
    maybeNegative = (-1)**ZZ.random_element(2)
    rational = maybeNegative * rational
    return rational

#No restrictions on coefficients or numerators
def createThreeRandomIntegers():
    a = maybeMakeNegative(ZZ.random_element(3, 9))
    b = maybeMakeNegative(ZZ.random_element(3, 9))
    c = maybeMakeNegative(ZZ.random_element(3, 9))
    return [a, b, c]

def createThreeDistinctRandomNaturals():
    possibleNaturals= range(2,7)
    n1, n2, n3 = sample(possibleNaturals, 3)
    naturals = [Integer(n1), Integer(n2), Integer(n3)]
    return naturals

def createThreeDistinctRandomIntegers():
    a, b, c = sample(range(3, 8), 3)
    return [Integer(maybeMakeNegative(a)), Integer(maybeMakeNegative(b)), Integer(maybeMakeNegative(c))]

def createViableConstants():
    a, b, c = createThreeRandomIntegers()
    d, e, f = createThreeRandomIntegers()
    g, h, i = createThreeDistinctRandomNaturals()
    # Check that there is exactly one solution to the linear equation
    OneSolutionCheck = (a/g) - (b/h) - (c/i)
    while (OneSolutionCheck == 0):
        a, b, c = createThreeRandomIntegers()
        d, e, f = createThreeRandomIntegers()
        g, h, i = createThreeDistinctRandomNaturals()
        OneSolutionCheck = (a/g) - (b/h) - (c/i)
    return [a, b, c, d, e, f, g, h, i]

def createSolution(constants):
    a, b, c, d, e, f, g, h, i = constants
    equationBlockOne = (a*x+d)/g
    equationBlockTwo = (b*x+e)/h
    equationBlockThree = (c*x+f)/i
    toSolve = equationBlockOne - equationBlockTwo - equationBlockThree
    solution = round(float(solve(toSolve, x)[0].rhs()), 3)
    return solution

######## END OF DEFINITIONS ###########

##### QUESTION 13 #####
constants13 = createViableConstants()
displayNumeratorA13 = constants13[0] * x + constants13[3]
displayNumeratorB13 = constants13[1] * x + constants13[4]
displayNumeratorC13 = constants13[2] * x + constants13[5]
answer13 = createSolution(constants13)

##### QUESTION 14 #####
constants14 = createViableConstants()
displayNumeratorA14 = constants14[0] * x + constants14[3]
displayNumeratorB14 = constants14[1] * x + constants14[4]
displayNumeratorC14 = constants14[2] * x + constants14[5]
answer14 = createSolution(constants14)

##### QUESTION 15 #####
constants15 = createViableConstants()
displayNumeratorA15 = constants15[0] * x + constants15[3]
displayNumeratorB15 = constants15[1] * x + constants15[4]
displayNumeratorC15 = constants15[2] * x + constants15[5]
answer15 = createSolution(constants15)
\end{sagesilent}

\begin{question}
Solve the equation below. 

$\frac{\sage{displayNumeratorA13}}{\sage{constants13[6]}} - \frac{\sage{displayNumeratorB13}}{\sage{constants13[7]}} = \frac{\sage{displayNumeratorC13}}{\sage{constants13[8]}}$

\begin{hint}
Adding/Multiplying fractions can be difficult and tedious. Is there something we can multiply both sides of the equation by to remove the fractions from the equation?
\end{hint}

$x = \answer[tolerance=0.05]{\sage{answer13}}$
\end{question}

\begin{question}
Solve the equation below. 

$\frac{\sage{displayNumeratorA14}}{\sage{constants14[6]}} - \frac{\sage{displayNumeratorB14}}{\sage{constants14[7]}} = \frac{\sage{displayNumeratorC14}}{\sage{constants14[8]}}$

$x = \answer[tolerance=0.05]{\sage{answer14}}$
\end{question}

\begin{question}
Solve the equation below. 

$\frac{\sage{displayNumeratorA15}}{\sage{constants15[6]}} - \frac{\sage{displayNumeratorB15}}{\sage{constants15[7]}} = \frac{\sage{displayNumeratorC15}}{\sage{constants15[8]}}$

$x= \answer[tolerance=0.5]{\sage{answer15}}$
\end{question}

\end{document}