\documentclass{ximera}
%\usepackage{sagetex}
%% handout
%% space
%% newpage
%% numbers
%% nooutcomes

%%% You can put user macros here
%% However, you cannot make new environments

\graphicspath{{./}{module1Activity/}{module2Activity/}{module3Activity/}}

\usepackage{sagetex}
\usepackage{tikz}
\usepackage{hyperref}
\usepackage{tkz-euclide}
\usetkzobj{all}
\pgfplotsset{compat=1.7} % prevents compile error.

\tikzstyle geometryDiagrams=[ultra thick,color=blue!50!black]
 %% we can turn off input when making a master document

\outcome{Understand the different sets of numbers along with the properties of these sets.}
\author{Darryl Chamberlain Jr.}
 
\title{Order of Operations}

\begin{document}
\begin{abstract}
Apply the properties of Real numbers to simplify expressions. 
\end{abstract}
\maketitle

\href{https://cnx.org/contents/mwjClAV_@8.1:0KhpF2RH@23/Real-Numbers-Algebra-Essentials}{Link to section in textbook}

%%%%%%%%%%%%%%%%%%%%%
%%%  Objective 3  %%%
%%%%%%%%%%%%%%%%%%%%%

Now that we have the terminology for the different sets of numbers, we can review their properties. We'll start with the Real numbers first. Watch \href{https://mediasite.video.ufl.edu/Mediasite/Play/adc9e0878901450db891a5b46b383bf01d}{this video} to review the properties of Real numbers. \textit{Note: You won't be asked to define a property or know the property by name. However, you \textbf{will} need to know how to use the properties to simplify in order.}

We'll focus on Order of Operations here as many students were taught an order that \textbf{does not align with how most calculators/computers simplify expressions.}

\begin{question}
Fill in the order below.

P: $\answer[format=string]{Parentheses}$

E: $\answer[format=string]{Exponents}$

M/D: $\answer[format=string]{Multiplication}$ \ $\answer[format=string]{Division}$

A/S: $\answer[format=string]{Addition}$ \ $\answer[format=string]{Subtraction}$
\end{question}

\begin{question}
Let's take a closer look at why M/D is written on the same level. 
$7 \div 5 \times 4 = \answer{5.6}$

$7 \times \frac{1}{5} \times 4 = \answer{5.6}$

Multiplying by $\frac{1}{5}$ is the same as dividing by $\answer{5}$. Now let's see what happens if we did multiplication first. 

$7 \div (5 \times 4) = \answer{0.35}$

By changing everything to multiplication, we can see why it is so important to read from left-to-right when operations are on the same level! 
\end{question}

Now try to simplify the more complicated expressions below.  

\begin{sagesilent}
# Order of Operations Question
# Structure:
    # $\sage{constants[0]} - \sage{constants[1]} \div \sage{constants[2]} * \sage{constants[3]} - (\sage{constants[4]} * \sage{constants[5]})$

def convertToSage(constants):
    p0, p1, p2, p3, p4, p5 = constants
    c0 = Integer(p0)
    c1 = Integer(p1)
    c2 = Integer(p2)
    c3 = Integer(p3)
    c4 = Integer(p4)
    c5 = Integer(p5)
    return [c0, c1, c2, c3, c4, c5]

def generateSolution(constants):
    c0, c1, c2, c3, c4, c5 = convertToSage(constants)
    solution = float((c0 - ((c1/c2) * c3)) - (c4 * c5 ))
    print "Solution: %f" %solution
    return solution

def generateDistractor(constants):
    c0, c1, c2, c3, c4, c5 = convertToSage(constants)
    solution = float(c0 - (c1/(c2 * c3)) - (c4 * c5 ))
    print "Distractor: %f" %solution
    return solution

def createConstants():
    listNaturals=list(range(2, 21))
    # Array of 6 distinct integers
    c0, c1, c2, c3, c4, c5 = convertToSage(sample(listNaturals, 6))
    constants = [c0, c1, c2, c3, c4, c5]
    solution = generateSolution(constants)
    distractor = generateDistractor(constants)
    # CHECKS if doing the question wrong will still give the correct solution.
    index = 0
    while solution == distractor:
        listNaturals = list(range(2, 21))
        c0, c1, c2, c3, c4, c5 = convertToSage(sample(listNaturals, 6))
        constants = [c0, c1, c2, c3, c4, c5]
        solution = generateSolution(constants)
        distractor = generateDistractor(constants)
        # Makes sure we don't get stuck in an infinite loop
        index += 1
        if (index > 100):
            break
        print index
    print solution, distractor
    return constants

########## END OF DEFINITIONS ###########

######### QUESTION 7 ##########
constants7 = createConstants()
answer7 = generateSolution(constants7)

######### QUESTION 8 ##########
constants8 = createConstants()
answer8 = generateSolution(constants8)

######### QUESTION 9 ##########
constants9 = createConstants()
answer9 = generateSolution(constants9)

\end{sagesilent}

\begin{question}
Simplify the expression $\sage{constants7[0]} - \sage{constants7[1]} \div \sage{constants7[2]} * \sage{constants7[3]} - (\sage{constants7[4]} * \sage{constants7[5]})$. 

$\answer[tolerance=0.05]{\sage{answer7}}$
\end{question}

\begin{question}
	Simplify the expression $\sage{constants8[0]} - \sage{constants8[1]} \div \sage{constants8[2]} * \sage{constants8[3]} - (\sage{constants8[4]} * \sage{constants8[5]})$. 
	
	$\answer[tolerance=0.05]{\sage{answer8}}$
\end{question}

\begin{question}
	Simplify the expression $\sage{constants9[0]} - \sage{constants9[1]} \div \sage{constants9[2]} * \sage{constants9[3]} - (\sage{constants9[4]} * \sage{constants9[5]})$.
	
	$\answer[tolerance=0.05]{\sage{answer9}}$
\end{question}

\end{document}
