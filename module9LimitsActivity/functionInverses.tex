\documentclass{ximera}
\usepackage{sagetex}
%% handout
%% space
%% newpage
%% numbers
%% nooutcomes
 
%% You can put user macros here
%% However, you cannot make new environments

\graphicspath{{./}{module1Activity/}{module2Activity/}{module3Activity/}}

\usepackage{sagetex}
\usepackage{tikz}
\usepackage{hyperref}
\usepackage{tkz-euclide}
\usetkzobj{all}
\pgfplotsset{compat=1.7} % prevents compile error.

\tikzstyle geometryDiagrams=[ultra thick,color=blue!50!black]
 %% we can turn off input when making a master document
 
\outcome{}
\author{Darryl Chamberlain Jr.}
  
\title{Objective 4 - Inverse}
 
\begin{document}
\begin{abstract}
Find the inverse of a function, if it exists.
\end{abstract}
\maketitle
 
\href{https://cnx.org/contents/mwjClAV_@8.1:9ZKq0BnY@15/Inverse-Functions}{Link to section in online textbook}
 
%%%%%%%%%%%%%%%%%%%%%
%%%  Objective 4  %%%
%%%%%%%%%%%%%%%%%%%%%
 
First, watch
% UPDATE VIDEO LINK
\underline{\href{https://mediasite.video.ufl.edu/Mediasite/Play/b9f8a4e77dc54b7ba6592d429c10b4911d}{this video}} to learn when a function has an inverse and how to find the inverse of a function.
% OBJECTIVE of video.
Feel free to pause the video and fill out the notes as you go.
 
% Find the inverse of the function, if it exists. Then, provide the interval the inverse is defined for.
 
\begin{sagesilent}
x = var("x")
 
def maybeMakeNegative(natural):
    integer = natural*(-1)**ZZ.random_element(2)
    return integer
 
def generateInverseLinear():
    a = maybeMakeNegative(ZZ.random_element(2, 8))
    b = maybeMakeNegative(ZZ.random_element(2, 8))
    line1 = a * x + b
    line2 = (x - b)/a
    return [line1, line2]
 
def generateInverseSquareRoot():
    h1 = maybeMakeNegative(ZZ.random_element(2, 8))
    h2 = maybeMakeNegative(ZZ.random_element(2, 8))
    k = maybeMakeNegative(ZZ.random_element(2, 8))
    squareRoot = sqrt(h1*x + h2) + k
    pivot = -h2/h1
    if (h1*(pivot - 1) + h2) &lt; 0:
        domain = [pivot, Infinity]
    else:
        domain = [-Infinity, pivot]
    squareRootInverse = ((x-k)**2 - h2)/h1
    return [squareRoot, squareRootInverse, domain]
 
def generateSquared():
    h1 = maybeMakeNegative(ZZ.random_element(2, 8))
    h2 = maybeMakeNegative(ZZ.random_element(2, 8))
    k = maybeMakeNegative(ZZ.random_element(2, 8))
    squared = (h1*x + h2)**2 + k
    return squared
 
def generateInverseCube():
    h1 = maybeMakeNegative(ZZ.random_element(2, 8))
    h2 = maybeMakeNegative(ZZ.random_element(2, 8))
    k = maybeMakeNegative(ZZ.random_element(2, 8))
    cube = (h1*x + h2)**3 + k
    cubeInverse = ((x-k)**(1/3) - h2)/h1
    return [cube, cubeInverse]
 
def generateInverseCubeRoot():
    h1 = maybeMakeNegative(ZZ.random_element(2, 8))
    h2 = maybeMakeNegative(ZZ.random_element(2, 8))
    k = maybeMakeNegative(ZZ.random_element(2, 8))
    cubeRoot = (h1*x + h2)**(1/3) + k
    cubeRootInverse = ((x-k)**3 - h2)/h1
    return [cubeRoot, cubeRootInverse]
#########
f1, f1Inverse = generateInverseCube()
f2 = generateSquared()
f3, f3Inverse = generateInverseLinear()
f4, f4Inverse, domain4 = generateInverseSquareRoot()
while domain4[0] > -Infinity:
    f4, f4Inverse, domain4 = generateInverseSquareRoot()
f5, f5Inverse = generateInverseCubeRoot()
 
\end{sagesilent}
 
% Q1 - Inverse for cubic.
\begin{question}
Determine whether the function below is 1-1.
 
$$ f(x) = \sage{f1} $$
 
$\answer[format=string]{Yes}$
 
\begin{feedback}
``Yes" or ``No".
\end{feedback}
 
If $f(x)$ is 1-1, find the inverse and define the domain on which the inverse is valid. If $f(x)$ is not 1-1, put ``NA" for all answer blocks.
 
$$ f^{-1}(x) = \answer{\sage{f1Inverse}} $$
 
\begin{feedback}
To find the inverse of a function, switch $x$ and $y$, then solve for $y$. Don't round.
\end{feedback}
 
Domain of $f^{-1}(x)$: $\answer[format=string]{(} \answer{\sage{-Infinity}}, \answer{\sage{Infinity}} \answer[format=string]{)}$
 
\begin{hint}
Think about the shape of the original function: are there places whether the function is not defined?
\end{hint}
 
\end{question}
 
% Q2 - Inverse for squared - DNE.
\begin{question}
 
Determine whether the function below is 1-1.
 
$$ f(x) = \sage{f2} $$
 
$\answer[format=string]{No}$
 
\begin{feedback}
``Yes" or ``No".
\end{feedback}
 
If $f(x)$ is 1-1, find the inverse and define the domain on which the inverse is valid. If $f(x)$ is not 1-1, put ``NA" for all answer blocks.
 
$$ f^{-1}(x) = \answer[format=string]{NA} $$
 
\begin{feedback}
To find the inverse of a function, switch $x$ and $y$, then solve for $y$. Don't round.
\end{feedback}
 
Domain of $f^{-1}(x)$: $\answer[format=string]{NA} \answer[format=string]{NA}, \answer[format=string]{NA} \answer[format=string]{NA}$
 
\begin{hint}
Think about the shape of the original function: are there places whether the function is not defined?
\end{hint}
 
\end{question}
 
% Q3 - Inverse for linear.
\begin{question}
Determine whether the function below is 1-1.
 
$$ f(x) = \sage{f3} $$
 
$\answer[format=string]{Yes}$
 
\begin{feedback}
``Yes" or ``No".
\end{feedback}
 
If $f(x)$ is 1-1, find the inverse and define the domain on which the inverse is valid. If $f(x)$ is not 1-1, put ``NA" for all answer blocks.
 
$$ f^{-1}(x) = \answer{\sage{f3Inverse}} $$
 
\begin{feedback}
To find the inverse of a function, switch $x$ and $y$, then solve for $y$. Don't round.
\end{feedback}
 
Domain of $f^{-1}(x)$: $\answer[format=string]{(} \answer{\sage{-Infinity}}, \answer{\sage{Infinity}} \answer[format=string]{)}$
 
\begin{hint}
Think about the shape of the original function: are there places whether the function is not defined?
\end{hint}
 
\end{question}
 
% Q4 - Inverse for square root.
\begin{question}
Determine whether the function below is 1-1.
 
$$ f(x) = \sage{f4} $$
 
$\answer[format=string]{Yes}$
 
\begin{feedback}
``Yes" or ``No".
\end{feedback}
 
If $f(x)$ is 1-1, find the inverse and define the domain on which the inverse is valid. If $f(x)$ is not 1-1, put ``NA" for all answer blocks.
 
$$ f^{-1}(x) = \answer{\sage{f4Inverse}} $$
 
\begin{feedback}
To find the inverse of a function, switch $x$ and $y$, then solve for $y$. Don't round.
\end{feedback}
 
Domain of $f^{-1}(x)$: $\answer[format=string]{(} \answer{\sage{-Infinity}}, \answer{\sage{domain4[1]}} \answer[format=string]{]}$
 
\begin{hint}
Think about the shape of the original function: are there places whether the function is not defined?
\end{hint}
 
\end{question}
 
% Q5 - Inverse for cube root.
\begin{question}
Determine whether the function below is 1-1.
 
$$ f(x) = \sage{f5} $$
 
$\answer[format=string]{Yes}$
 
\begin{feedback}
``Yes" or ``No".
\end{feedback}
 
If $f(x)$ is 1-1, find the inverse and define the domain on which the inverse is valid. If $f(x)$ is not 1-1, put ``NA" for all answer blocks.
 
$$ f^{-1}(x) = \answer{\sage{f5Inverse}} $$
 
\begin{feedback}
To find the inverse of a function, switch $x$ and $y$, then solve for $y$. Don't round.
\end{feedback}
 
Domain of $f^{-1}(x)$: $\answer[format=string]{(} \answer{\sage{-Infinity}}, \answer{\sage{Infinity}} \answer[format=string]{)}$
 
\begin{hint}
Think about the shape of the original function: are there places whether the function is not defined?
\end{hint}
\end{question}
 
\end{document}