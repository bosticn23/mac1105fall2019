\documentclass{ximera}
%\usepackage{sagetex}
%% handout
%% space
%% newpage
%% numbers
%% nooutcomes

%%% You can put user macros here
%% However, you cannot make new environments

\graphicspath{{./}{module1Activity/}{module2Activity/}{module3Activity/}}

\usepackage{sagetex}
\usepackage{tikz}
\usepackage{hyperref}
\usepackage{tkz-euclide}
\usetkzobj{all}
\pgfplotsset{compat=1.7} % prevents compile error.

\tikzstyle geometryDiagrams=[ultra thick,color=blue!50!black]
 %% we can turn off input when making a master document

\outcome{}
\author{Darryl Chamberlain Jr.}
 
\title{Objectives}

\begin{document}
\begin{abstract}
List of objectives for Module 9L - Operations on Functions.
\end{abstract}
\maketitle

% Introduction to the section. 
\textbf{Order of Operations} - this tells us the order that we operate on numbers. This could be operating on a single number (e.g., $3^2$, where ``squared" is operating on the number $3$) or between two numbers (e.g., $3*2$, where ``multiplying" is operating on the numbers $3$ and $2$). But we don't have to \textbf{just} operate on numbers! In fact, we've been operating on functions since Module 4 - Quadratic Functions. When we thought about how to graph a quadratic function, we could think about the parent function, $x^2$, and think about how the vertex is shifted and if the graph should be flipped (based on the leading coefficient). Shifting and flipping a function are ways we can operate on a function - we take a function and change it in a prescribed way. This Module will work through other ways we can operate on functions: adding/subtracting/multiplying/dividing, composing, and inverting. 

% How does it relate to previous section?

% How will it be useful for the future?

The objectives for this Module are: 
\begin{enumerate}
    \item Identify the \href{https://cnx.org/contents/mwjClAV_@8.1:ik_Ed0Pa@12/Composition-of-Functions}{domain after operating} $(+, -, \text{x}, \div)$ on functions.
    \item \href{https://cnx.org/contents/mwjClAV_@8.1:ik_Ed0Pa@12/Composition-of-Functions}{Evaluate the composition of two functions.}
    \item \href{https://cnx.org/contents/mwjClAV_@8.1:9ZKq0BnY@15/Inverse-Functions}{Determine whether a function is 1-1.}
    \item \href{https://cnx.org/contents/mwjClAV_@8.1:9ZKq0BnY@15/Inverse-Functions}{Find the inverse of a function, if it exists.}
\end{enumerate}

\end{document}