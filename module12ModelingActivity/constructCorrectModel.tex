\documentclass{ximera}
\usepackage{sagetex}
%% handout
%% space
%% newpage
%% numbers
%% nooutcomes
 
%% You can put user macros here
%% However, you cannot make new environments

\graphicspath{{./}{module1Activity/}{module2Activity/}{module3Activity/}}

\usepackage{sagetex}
\usepackage{tikz}
\usepackage{hyperref}
\usepackage{tkz-euclide}
\usetkzobj{all}
\pgfplotsset{compat=1.7} % prevents compile error.

\tikzstyle geometryDiagrams=[ultra thick,color=blue!50!black]
 %% we can turn off input when making a master document
 
\outcome{}
\author{Darryl Chamberlain Jr.}
  
\title{Objective 2 - Construct Correct Model}
 
\begin{document}
\begin{abstract}

\end{abstract}

\maketitle
 
\textit{Note: There are no textbook or videos directly to this section. If you want to review a certain type of model, you will need to go back to that Module.}
 
%%%%%%%%%%%%%%%%%%%%%
%%%  Objective 1  %%%
%%%%%%%%%%%%%%%%%%%%%

\textbf{General tips to constructing a model:}
\begin{itemize}
\item Identify the appropriate function to model the situation.
\item Try introducing small numbers to check your model. \textit{For example, if you need to model population growth, try using a small population like 10 to make sure you are seeing the growth you expect.}
\item Check your units and variables. 
\end{itemize}

% LINEAR
\begin{sagesilent}
v = var('v')
concA = ZZ.random_element(1, 4)*5
concB = ZZ.random_element(2, 5)*10
concTotal = ZZ.random_element(10, 31)
while concTotal < concA or concTotal > concB:
    concTotal = ZZ.random_element(10, 31)
totalVolume = ZZ.random_element(5, 15)
eqPartA6 = totalVolume - v
eqPartB6 = concA * 0.01 * v
eqPartC6 = concB * 0.01 * eqPartA6
eqPartD6 = eqPartB6 + eqPartC6
\end{sagesilent}
\begin{question}
Chemists commonly create a solution by mixing two products of differing concentrations together. For example, a chemist could have large amounts of a $\sage{concA}\%$ acid solution and a $\sage{concB}\%$ acid solution, but need a $\sage{totalVolume}$ liter $\sage{concTotal}$\% solution. Construct a model that describes the amount of acid in a $\sage{concTotal}\%$ acid solution, $A_{\sage{concTotal}}$, in terms of the volume of the $\sage{concA}\%$ acid solution, $v$. 

$A_{\sage{concTotal}} = \answer{\sage{eqPartD6}}$

\end{question}

% EXPONENTIAL 
\begin{sagesilent}
t = var('t')
halfLifeYears2 = ZZ.random_element(100, 1001)*ZZ.random_element(100, 1001)
initialAmount2 = ZZ.random_element(100, 1000)
k2 = -ln(2)/halfLifeYears2
equation2 = initialAmount2*e**(k2*t)
\end{sagesilent}
\begin{question}
There is initially $\sage{initialAmount2}$ grams of element $X$. The half-life of element $X$ is $\sage{halfLifeYears2}$ years. Describe the amount of element $X$ remaining as a function of time, $t$, in years.

$X(t) = \answer{\sage{initialAmount2}} \answer{e}^{\answer{\sage{k2}} t}$
\end{question}

% LINEAR
\begin{sagesilent}
x = var("x")
fixedCost1 = ZZ.random_element(10, 26)*1000
productionCost1 = round(ZZ.random_element(1, 4)*0.05, 2)
sellingPoint1 = round(productionCost1*ZZ.random_element(1, 4), 2)
costs1 = productionCost1*x + fixedCost1
profits1 = sellingPoint1*x
revenue1 = profits1 - costs1
\end{sagesilent}

\begin{question}
A company sells doughnuts. They incur a fixed cost of \$$\sage{fixedCost1}$ for rent, insurance, and other expenses. It costs \$$\sage{productionCost1}$ to produce each doughnut. The company sells each doughnut for \$$\sage{sellingPoint1}$. Construct a model that describes their total revenue, $R$, as a function of the number of doughnuts, $x$, they produce.

$R(x) = \answer{\sage{revenue1}}$

\end{question}

% DIRECT 
\begin{sagesilent}
k = var('k')
a = var('a')
k4 = round((4*pi**2) / (9.8), 3)
print "%s is this thing here" %k4
a4 = ZZ.random_element(3, 9)
T4 = round(k4*sqrt(a4**3)*12, 2)
\end{sagesilent}

\begin{question}
Kepler's Third Law: The square of the time, $T$, required for a planet to orbit the Sun is directly proportional to the cube of the mean distance, $a$, that the planet is from the Sun. Assume that Mars' mean distance from the Sun is $\sage{a4}$ AUs and it takes Mars about $\sage{T4}$ months to orbit the Sun. Write the equation that describes time $T$ (years) in terms of the mean distance, $a$ (AUs).

$T(a) = \answer[tolerance=0.05]{\sage{k4}} a^{\answer{3/2}}$

\end{question}

\begin{sagesilent}
t = var('t')
r = var('r')
k5 = ln(0.5)/5730
equation5A = e**(t*k5)
equation5B = ln(r)/k5
percent5 = ZZ.random_element(1, 101)
old5 = round(equation5B(percent5*0.01), 0)
\end{sagesilent}

\begin{question}
The half-life of carbon-14 is 5,730 years. Describe the age in years of an object in terms of the ratio of carbon-14, $r = \frac{C}{C_0}$, remaining.

$t(r) = \answer[tolerance=0.01]{\sage{1/k5}} \ln(\answer{r}) $

\end{question}

% LINEAR
\begin{sagesilent}
m = var('m')
officerAspeed3 = ZZ.random_element(2, 7)
officerBspeed3 = ZZ.random_element(2, 7)
while officerAspeed3 == officerBspeed3:
    officerAspeed3 = ZZ.random_element(2, 7)
    officerBspeed3 = ZZ.random_element(2, 7)
partA3 = officerAspeed3/60 * m
partB3 = officerAspeed3/60 * m + officerBspeed3/60 * m
partC3 = sqrt((officerAspeed3/60)**2 + (officerBspeed3/60)**2)*m
\end{sagesilent}

\begin{question}
Two UFPD are patrolling the campus on foot. To cover more ground, they split up and begin walking in different directions. Office A is walking at $\sage{officerAspeed3}$ mph while Office B is walking at $\sage{officerBspeed3}$ mph. Construct a model that describes their total distance from each other, $T_2$, as a function of minutes, $m$, that have passed \textit{if they were walking in exactly 90 degrees from each other} (e.g., North/East).

$T_2(m) = \answer{\sage{partC3}}$

\end{question}

\begin{sagesilent}
t = var('t')
rateList1 = ["doubles", "triples", "quadruples"]
choose8 = ZZ.random_element(0, 3)
rate8 = rateList1[choose8]
initialBacteria8 = ZZ.random_element(1, 9)*100
k8 = ln(choose8+2)
bacteriaPopulation8 = initialBacteria8 * (choose8 + 2)**(k8*t)
\end{sagesilent}
\begin{question}
A population of bacteria $\sage{rate8}$ every hours. If the culture started with $\sage{initialBacteria8}$, write the equation that models the bacteria population after $t$ hours. 

$P(t) = \answer{\sage{initialBacteria8}} \answer{\sage{choose8+2}}^{\answer{t}}$

\end{question}

\end{document}